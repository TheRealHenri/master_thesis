\chapter{Conclusion\label{cha:chapter7}}
In this thesis, we set out to investigate techniques that can be effectively employed to ensure privacy and security measures, such as access control and data masking, in distributed event stores. We evaluated the effectiveness of these techniques by quantifying the performance overhead incurred due to the implementation of additional privacy and security measures. The aim was to bridge the gap in missing underlying infrastructure in distributed event stores to adequately support stringent data privacy regulations by governments worldwide \cite{Colombo2015}. Furthermore, we investigated the intersection of database access control and privacy, an area that has previously been explored by \cite{chaudhuri2011database}. \par
We presented a detailed model incorporating \ac{RBAC} and anonymization techniques in tandem within distributed event stores. We applied this model to Apache Kafka and built the abstraction of \acp{ACL} to \ac{RBAC} as well as the \acf{DASH}. \ac{DASH} is the first system that applies anonymization to data streams in real time for Apache Kafka. By coupling \ac{RBAC} with anonymized access control, our system achieved for the first time anonymization granularity for Apache Kafka. \par
With a comprehensive data pipeline, incorporating Kafka and its administrative component Zookeeper, deployable across various environments via Docker, we tested anonymization techniques and their impact on performance. These tests were performed using a mock dataset hosted on the \ac{DIMA} chair's server. \par
We found that our categorization of anonymization techniques based on the scope of operation translates well to their performance impact. Our experiments showed that value-based anonymization techniques have a negligible impact on performance. Even concatenating multiple anonymization techniques neither negatively impacted latency nor throughput significantly. Similar results were observed for tuple-based 
For attribute-based anonymization techniques, our experiments showed that an upper bound exists for throughput scaling with available resources. Up until that point, no performance decreases were detected, neither in latency nor throughput. Throughput exceeding that threshold, however, suffered significant performance decreases as the time-based windowing technique, that Kafka inherently provides, could not cope with the amount of data per window. A similar observation was made for table-based anonymization techniques. Our implemented k-anonymization was unable to exceed a fixed amount of throughput based on resources. Again for any lower amount of incoming throughput, the performance was not impacted. We attribute this to the computational requirements of the CASTLE algorithms. \par
Independently of the anonymization technique employed, consideration must be given to the additional storage costs. Multiplying data streams necessitates increased storage. \par 
In summary, we showcased a step towards enhancing data privacy in distributed event stores, through the integration of \ac{RBAC} and anonymization techniques creating granular anonymized access control. While acknowledging the challenges and complexities of modern database systems, our work demonstrated a viable pathway for achieving robust data security without compromising performance for masking functions with tuple-at-the-time semantics. As the landscape of data privacy continues to evolve, we believe the insights and methodologies developed during this thesis will contribute to more secure and privacy-conscious data management practices. 

\section{Future Work}
We believe our work creates the foundation of privacy-aware infrastructure within \ac{DES}. From here on out it facilitates the venture into multiple different directions. In the following, we provide some suggestions for potential future work:

\begin{enumerate}
\item \textbf{Graphical User Interface}\\
Adding a graphical user interface for the data officer holds the potential of improving the user experience threefold: First, it would lead to a more fault-tolerant and user-friendly configuration process as it abstracts away the necessity of writing the JSON file by hand. Second, it would simplify the process to no longer require a user with significant technical expertise when the configuration as well as the shell scripts become executable in a UI. This would open up the role of Data Officer to a broader user base. Third, it would truly unify the \ac{RBAC} system with \ac{DASH} as setting up the requirements for the anonymized streams could include the appropriate role-based access control setup. Both would then be achieved simultaneously under the hood.
\item \textbf{Expanding on the Adaptability of DASH}\\
We have developed \ac{DASH} exclusively for Apache Kafka, however, adaptability was always at the back of our mind. It would be interesting to see how \ac{DASH} applies to other Distributed Event Stores, and if our findings for the anonymization techniques translate there as well. As anonymization techniques continue to evolve, it would be intriguing to observe their application to \ac{DASH} and the data streaming realm.
\item \textbf{Parameter Optimization}\\
We found that the choice of parameters had a significant impact on the performance overhead of \ac{DASH}. We believe there is an optimum to be found for a given set of anonymization techniques and data. Taking into consideration information loss, performance overhead, and anonymization robustness, a neural net could be constructed to find the optimal configuration and provide this information to the user.
\end{enumerate}
