\chapter{Introduction\label{cha:chapter1}}
Distributed event stores capture, store, and process real-time data streams in distributed environments. They have been widely adopted across various sectors, including Fortune 100 companies, governments, healthcare, and transportation industries \cite{kafka} for their efficient, scalable, and fault-tolerant system design. In the modern data-driven world, the growing demand for comprehensive data privacy policies has resulted in increasingly stringent regulations from governments worldwide \cite{GDPR, CCPA}. However, the underlying infrastructure for distributed event stores to adequately support these policies is markedly lacking \cite{Colombo2015}. This disparity poses a unique challenge, particularly when considering the demands of modern database systems to maintain high performance, characterized by low latency and high throughput.\par
The present work aims to examine what techniques can be effectively employed to ensure privacy and security measures, such as access control and data masking, in distributed event stores. Additionally, it aims to explore what strategies can be developed and implemented to ensure that these privacy and security measures have a minimal impact on the performance of distributed event stores. \par
While there exists a body of work focusing on anonymization and data masking for data streaming \cite{Cao2008, KIDS_zhang, anonymizing_IoT}, there is a noticeable gap in research specifically targeting distributed event stores. 
Furthermore, although there are enterprise technologies for managing data flowing into such systems \cite{privitar}, there is limited literature on techniques designed for data already within distributed event stores. Most notably, the concept of integrating \ac{RBAC} within this framework, where the role assigned determines the level of anonymity accorded to the data, is a completely novel approach. \par
The introduction of access control coupled with anonymization in distributed event stores holds the potential to contribute to more advanced, efficient, and secure data handling. By making such tools accessible and cost-effective, companies might be more inclined to prioritize and invest in user data privacy.\par
The overarching goal of this thesis is to design, implement, and evaluate a management framework for distributed event stores. This framework aims to incorporate \ac{RBAC}, granting administrators the capacity to define levels of data anonymization and specify masking functions. By associating levels of anonymization with the number of additional data streams per original data stream, the framework will facilitate nuanced and customizable data privacy measures. Moreover, this plugin will facilitate the assignment of consumers to streams based on their specific roles, further enhancing granular control over data access and privacy within distributed event stores.\par
The scope of this thesis will primarily encompass the exploration of privacy and security measures in the context of distributed event stores. The thesis will delve into the administrative aspects of these technologies, including the implementation and management of access control and data masking functions. Particular attention will be paid to the role of administrators and the impact of their decisions on the performance of these systems. This includes the computational implications of various anonymization techniques. It will involve an in-depth study of different strategies' performance impacts, aiming to minimize overhead while maintaining robust data privacy. At the same time, this thesis will not specifically address or align with any particular data privacy policy such as the \ac{GDPR} \cite{GDPR}. While the thesis is designed around general principles of data privacy and security, it will not provide policy-specific solutions or address the nuances of any specific regulatory framework. Furthermore, the anonymization of data flowing into or out of distributed event stores is beyond the scope of this thesis. The focus will primarily be on the data within the systems, and not on the methods and protocols for handling data entering or leaving these systems. \par
This thesis introduces the \acf{DASH}, a system architected to process data streams into multiple anonymized versions. This design aims to provide users of distributed event stores with a nuanced granularity of anonymization. A comprehensive survey of anonymization techniques is presented, categorizing these methodologies for practical application in various scenarios. \ac{DASH} includes an extensive library of anonymization techniques, offering users the flexibility to tailor the anonymization process to their specific requirements. Additionally, role-based access control is made available as a separate component to assign anonymized versions to streams. This combination of access control with anonymization variety is not only theoretically robust but also practically applicable. Its synergy is highlighted in our theoretical framework chapter, where we showcase a real-world example. Furthermore, this thesis presents a theoretical model designed to simplify future implementations of anonymized access control across various distributed event stores. \par
In the context of this research, we have tested and evaluated \ac{DASH} in an extensive data pipeline environment. We have found \dots \par
The thesis is structured as follows: in Chapter \ref{cha:chapter2} we conduct a literature review. The theoretical framework is constructed in Chapter \ref{cha:chapter3}. Subsequently, Chapter \ref{cha:chapter4} examines in detail the implementation. Tests and their evaluations are addressed in Chapter \ref{cha:chapter5}. Related Work is the subject of Chapter \ref{cha:chapter6}. Finally, Chapter \ref{cha:chapter7} presents the conclusion and offers an outlook on future work.
