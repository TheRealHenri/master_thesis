\chapter{Introduction\label{cha:chapter1}}
\acfp{DES} capture, store, and process real-time data streams in distributed environments. They have been widely adopted across various sectors, including Fortune 100 companies, governments, healthcare, and transportation industries \cite{kafka} for their efficient, scalable, and fault-tolerant system design. In the modern data-driven world, the growing demand for comprehensive data privacy policies has resulted in increasingly stringent regulations from governments worldwide \cite{GDPR, CCPA}. However, the underlying infrastructure for \acp{DES} to adequately support these policies is markedly lacking \cite{Colombo2015}. This disparity poses a unique challenge, particularly when considering the demands of modern database systems to maintain high performance, as characterized by low latency and high throughput.\par
The present work aims to examine what techniques can be effectively employed to ensure privacy and security measures, such as access control and data masking, in \acp{DES}. Additionally, it aims to explore what strategies can be developed and implemented to ensure that these privacy and security measures have minimal impact on the performance of \acp{DES}. \par

Despite the evident effectiveness and efficiency of \acp{DES}, companies are hesitant to adopt this technology due to data privacy concerns \cite{Intel2012BigData}. This concern is also taken up in Colombo and Ferrari's journal article \cite{Colombo2015}, where the authors highlight the security risks of Big Data Platforms, which include \acp{DES}, and emphasize the need for robust access control and enhanced privacy protection. Although they propose a research roadmap to address these issues, a concrete solution is not yet provided. In \cite{chaudhuri2011database}, Chaudhuri et al. explore the intersection of database access control and privacy, offering solutions at the query processing level within database systems. While their focus is on relational databases, their work further underscores the broader need for enhanced privacy protection, a challenge also for the unstructured data in \acp{DES}. \par

While there exists a body of work focusing on anonymization and data masking for data streaming \cite{Cao2008, KIDS_zhang, anonymizing_IoT}, there is a noticeable gap in research specifically targeting \acp{DES}. 
Furthermore, although there are enterprise technologies for managing data flowing into such systems \cite{privitar}, there is limited literature on techniques designed for data already within \acp{DES}. \par

The overarching goal of this thesis is to design, implement, and evaluate a comprehensive management system for \acp{DES}. This system aims to incorporate data privacy policy enforcing mechanisms, thus granting administrators the capacity to define levels of data anonymization and specify masking functions. By associating levels of anonymization with the number of additional data streams per original data stream, the architecture will facilitate nuanced and customizable data privacy measures. Moreover, this system will enable the assignment of consumers to streams based on their specific roles, further enhancing granular control over data access and privacy within \acp{DES}. The introduction of access control coupled with anonymization in \acp{DES} holds the potential to contribute to more advanced, efficient, and secure data handling. By making such tools accessible and cost-effective, companies might be more inclined to prioritize and invest in user data privacy.\par

Integrating anonymized access control into \acp{DES} is challenging for various reasons: The existing infrastructure to support data privacy policy is essentially absent and has to be built from the ground up. Moreover, these solutions must ensure data anonymization without introducing significant performance overhead, given the strict performance requirements of \acp{DES}. Additionally, the design of these solutions should remain policy-agnostic, not aligning with specific data privacy regulations like the \ac{GDPR} \cite{GDPR}, to maintain broad applicability. Furthermore, the solution must be versatile enough to be customizable to fit a diverse range of applications. \par

This thesis introduces the \acf{DASH}, a system architected to process data streams into multiple anonymized versions. This design aims to provide users of \acp{DES} with a nuanced granularity of anonymization. A comprehensive survey of anonymization techniques is presented, categorizing these methodologies for practical application in various scenarios. \ac{DASH} includes an extensive library of anonymization techniques, offering users the flexibility to tailor the anonymization process to their specific requirements. Additionally, role-based access control is made available as a separate component to assign anonymized versions to streams. This combination of access control with anonymization variety is not only theoretically robust but also practically applicable. Its synergy is highlighted in our theoretical framework chapter, where we showcase a real-world example. Furthermore, this thesis presents a theoretical model designed to simplify future implementations of anonymized access control across various \acp{DES}. \par

In the context of this research, we have tested and evaluated \ac{DASH} in an extensive data pipeline environment. Our evaluation demonstrates that \ac{DASH} can anonymize incoming data efficiently, showing no significant performance overhead on a per-tuple basis, regardless of data. More demanding anonymization techniques operating on collections of tuples proved to be manageable by \ac{DASH} without impeding throughput up to a certain level. For attribute-based anonymization techniques, such as aggregation, we found that the windowing techniques, discretizing the unbounded data streams, are the bottleneck. Table-based anonymization techniques, such as those enforcing k-anonymity have proven to be exceptionally demanding in computational resources, and \ac{DASH} unfortunately is no exception. However, given enough resources \ac{DASH} can maintain the throughput without falling behind. In summary, \ac{DASH} has proven to be a versatile and high-performing solution for anonymized access control We see the potential for further work in applying our model to other \acp{DES} to verify the generalizability of our findings. The step would be the development of a graphical user interface for \ac{DASH} configuration that could enhance user-friendliness and accessibility for a wider audience. \par 

This thesis is structured as follows: Chapter \ref{cha:chapter2} gives an overview of the literature regarding \acp{DES}, access control, and anonymization techniques, specifically focusing on data streams. The theoretical framework for our management system is constructed in Chapter \ref{cha:chapter3}. Subsequently, Chapter \ref{cha:chapter4} examines in detail the implementations of the components of that system. We assess the capabilities of \ac{DASH} in \ref{cha:chapter5} by evaluating the tests we have run with the system. Related Work is the subject of Chapter \ref{cha:chapter6}. Finally, Chapter \ref{cha:chapter7} summarizes our main findings and offers an outlook on potential future work. 
