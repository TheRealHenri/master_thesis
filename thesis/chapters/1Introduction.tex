\chapter{Introduction\label{cha:chapter1}}

%This chapter should include the following sections.
... should include the following:
\begin{itemize}
\item motivation (why is this problem interesting? offer examples),
\item research challenge (what is the obstacle to be overcome?),
\item novelty (was this problem already solved?),
\item anticipated impact (how does solving this problem impact our world?).
\end{itemize}

\section{Motivation\label{sec:moti}}
%This section should 
%\begin{itemize}
 %   \item answer the question - why is this problem interesting? 
 %   \item offer examples illustrating the problem.
%\end{itemize}
The increasing popularity of data streaming in corporations highlights the imperative need for incorporating anonymization and data masking techniques in this technology. Particularly noteworthy is the extensive adoption of distributed event stores, which are scalable and fault-tolerant systems designed to capture, store, and process real-time data streams, across various sectors, including Fortune 100 companies, governments, healthcare, and transportation industries [5]. This widespread usage emphasizes the criticality of ensuring data privacy within these distributed event stores, while also highlighting the potential transformative impact of effective anonymization and data masking techniques in this domain.


\section{Research Challenge\label{sec:objective}}
%This section should answer the question -
%\begin{itemize}
 %   \item what is the obstacle to be overcome?
%\end{itemize}
In the contemporary data-driven world, the growing demand for comprehensive data privacy policies is matched by increasingly stringent regulations from governments worldwide [1,2]. However, the underlying infrastructure to adequately support these policies is markedly lacking [3,4]. This disparity poses a unique challenge, particularly when considering the demands of modern database systems to maintain high performance, characterized by low latency and high throughput.

\section{Novelty \label{sec:scope}}
%This section should answer the question -
%\begin{itemize}
  %  \item was this problem already solved?
%\end{itemize}
While there exists a body of work focusing on anonymization and data masking for data streaming [6,7,8], there is a noticeable gap of research specifically targeting distributed event stores. Furthermore, although there are enterprise technologies for managing data flowing into such systems [9], there is limited literature on techniques designed for data already within. Most notably, the concept of integrating Role-Based Access Control (RBAC) within this framework, where the role assigned determines the level of anonymity accorded to the data, is a completely novel approach.

\section{Anticipated Impact \label{sec:outline}}

%This section should answer the question -
%\begin{itemize}
 %   \item how does solving this problem impact our world?
%\end{itemize}

%Conclude this subsection with an image describing 'the big picture'. How does your solution fit into a larger environment? You may also add another image with the overall structure of your component.
The integration of data privacy policies with modern data stream structures, such as distributed event stores, would be a significant innovation. The introduction of Role-Based Access Control (RBAC) coupled with anonymization in distributed event stores holds the potential to contribute to more advanced, efficient, and secure data handling. By making such tools accessible and cost-effective, companies might be more inclined to prioritize and invest in user data privacy.

%'Figure \ref{fig:intro} shows Component X as part of ...' 
%\\
%\begin{figure}[htb]
%  \centering
%  \includegraphics[width=9cm]{intro_example.pdf}\\
%  \caption{Component X}\label{fig:intro}
%\end{figure}
\section{Research Problem }

... should include the following:
\begin{itemize}

\item a succinct, precise, and unambiguous statement of the research problem or question to be solved,
\item goals and subproblems that will be explored, including the scope of the thesis (i.e., what is in and out of scope).
\end{itemize}
\begin{comment} 
\begin{table}[h!]
\begin{center}
\begin{tabular}{ |c|c|c| } 
 \hline
 Column1 & Column2 & Column3 \\ [0.5ex] 
 \hline
 cell1 & cell2 & cell3 \\ 
 cell4 & cell5 & cell6 \\ 
 cell7 & cell8 & cell9 \\ 
 \hline
\end{tabular}
\caption{An Example Table.}
%\label{table:1}
\end{center}
\end{table}
\end{comment} 


\begin{table}[ht]
\begin{center}
\footnotesize{
\begin{tabular}{ c|c } 
 \hline
  Area (Million sq. miles) & Calling Code \\
  \hline
 0.29 & 56\\
 0.3 & 90\\
 3.8 & 1\\
 0.5 & 51\\
% \rowfont{\color{red}}
\textcolor{red}{600} & \textcolor{red}{9800}\\
 \hline
 \multicolumn{1}{c}{Pearson = 1.0} & \multicolumn{1}{c}{Spearman’s = 0.1}
\end{tabular}
}
\caption{Correlation in the existence of outlier\cite{EsmailoghliQZ21}.}
%\label{table:1}
\end{center}
\end{table}

Is there also indentation here? \\
Hard to tell without two lines\\ 
Oh wow there is 

\section{Outline}
% The 'structure' or 'outline' section gives a brief introduction into the main chapters of your work. Write 2-5 lines about each chapter. Usually diploma thesis are separated into 6-8 main chapters. 
% \\
% \\
% \noindent This example thesis is separated into 7 chapters.
% \\
% \\
% \textbf{Chapter \ref{cha:chapter2}} is usually termed 'Related Work', 'State of the Art' or 'Fundamentals'. Here you will describe relevant technologies and standards related to your topic. What did other scientists propose regarding your topic? This chapter makes about 20-30 percent of the complete thesis.
% \\
% \\
% \textbf{Chapter \ref{cha:chapter3}} analyzes the requirements for your component. This chapter will have 5-10 pages.
% \\
% \\
% \textbf{Chapter \ref{cha:chapter4}} is usually termed 'Concept', 'Design' or 'Model'. Here you describe your approach, give a high-level description to the architectural structure and to the single components that your solution consists of. Use structured images and UML diagrams for explanation. This chapter will have a volume of 20-30 percent of your thesis.
% \\
% \\
% \textbf{Chapter \ref{cha:chapter5}} describes the implementation part of your work. Don't explain every code detail but emphasize important aspects of your implementation. This chapter will have a volume of 15-20 percent of your thesis.
% \\
% \\
% \textbf{Chapter \ref{cha:chapter6}} is usually termed 'Evaluation' or 'Validation'. How did you test it? In which environment? How does it scale? Measurements, tests, screenshots. This chapter will have a volume of 10-15 percent of your thesis.
% \\
% \\
% \textbf{Chapter \ref{cha:chapter7}} summarizes the thesis, describes the problems that occurred and gives an outlook about future work. Should have about 4-6 pages.