\chapter{Literature Review\label{cha:chapter2}}

\section{Role Based Access Control}

\section{Distributed Event Stores}

\section{Anonymization\label{sec:anon}}


The concept of data anonymity plays a pivotal role in the context of access control in distributed event stores. One of the widely accepted models for preserving the privacy of data subjects in datasets is $K-Anonymity$. Introduced by Sweeney \cite{sweeney2002kanonymity}, the $K-Anonymity$ model posits that each data record within a dataset should be indistinguishable from at least $K-1$ other records with respect to any set of quasi-identifiers. Mathematically, a dataset adheres to $K-Anonymity$ if, for a set of quasi-identifiers $Q$, each sequence of values in $Q$ appears at least $K$ times in the dataset.


Techniques 
Cryptography, Masking Functions and there is a thirsd ?

\subsection{Principles}

\subsection{Data Streaming\label{lit:data_streaming}}
One prominent example of applying anonymization techniques to streamed data is the work CASTLE \cite{Cao2008}

Univariate Microaggregation \label{lit:univariate}

\section{Related Work}

... should include the following:
\begin{itemize}
    \item definitions / technical terms,
    \item theoretical foundations / principles,
    \item descriptions of algorithms, hardware, software, and/or systems employed.
\end{itemize}


% We can leave this out... The advisor can point this out, if it is a concern.
%Suggestion: Figures could be inserted in pdf form to avoid pixelation when the image is magnified.

%This section is intended to give an introduction about relevant terms, technologies and standards in the field of X. You do not have to explain common technologies such as HTML or XML. 


