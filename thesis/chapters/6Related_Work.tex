\chapter{Related Work\label{cha:chapter6}}
Organizations worldwide are increasingly forced by both government regulations and user demand to comply with data privacy and protection policies. This applies to all users of \acp{DES}, who are adapting their systems to comply with these international standards. \par
The most popular \acp{DES} Apache Kafka \cite{kafka}, Amazon Kinesis \cite{amazon_kinesis}, RabbitMQ \cite{rabbit_mq} and Apache Pulsar \cite{pulsar} differ only slightly in their security features. Table \ref{table:security_DES} shows a comparison.

\begin{table}[h]
    \centering
    \footnotesize
    \renewcommand{\arraystretch}{1.5}
    \begin{tabular}{|c|c|c|c|c|}
    \hline
    \textbf{\ac{DES}} & \textbf{Authentication} & \textbf{Authorization} & \textbf{Encryption} & \textbf{Anonymization} \\ \hline
    Kafka & SSL, SASL, Kerberos & \acp{ACL} & $\times$ &  $\times$ \\ \hline
    Kinesis & SSL, Multi-factor & \ac{RBAC} & \checkmark & $\times$ \\ \hline
    Pulsar & SSL, SASL, Kerberos & \ac{RBAC} & \checkmark & $\times$ \\ \hline
    RabbitMQ & SSL & \acp{ACL} & $\times$ & $\times$ \\ \hline
    \end{tabular}
    \caption{Security Features of the most popular \acp{DES} }
    \label{table:security_DES}
\end{table}

While all the mentioned \acp{DES} offer authentification and authorization measures, none of them provide anonymization. Notably, Kinesis and Pulsar at least already provide \ac{RBAC} and data encryption. Additionally, one thing holds true for all of them: Both authentification and authorization need to be specifically enabled. The default configurations contain no inherent data privacy measures. Their stance is clear: they leave the responsibility of enforcing data privacy regulations to their users. This leads to a patchwork of solutions varying from company to company. \par
A general approach is desirable for each \ac{DES} and its users. Data protection has become an attribute that sets one's business apart from competitors. It is part of the brand of a company as a selling point of its product. When deciding on a \ac{DES} for the development of a new product, the choice will be influenced by the inherent data protection of that \ac{DES}. Therefore, for a \ac{DES} to have integrated sufficient and efficient data protection in its infrastructure, could set them apart from the rest. Similarly, adhering to the government regulations lies in the responsibility of the users of the \ac{DES}. In any case, they have to integrate data protection into their workflows. An existing infrastructure would alleviate this development.\par
While there are many examples of individual solutions that attempt data protection- but most simply rely on encryption. Only one example comes close to the anonymization granularity we have presented in this thesis, and it is provided by Confluent \cite{confluent}: \par
Confluent turns Apache Kafka itself into a purchasable product in the form of a data streaming platform and cloud service. Founded by the creators of Apache Kafka and utilized by prominent companies such as BMW, Bosch, the Deutsche Bahn, and Domino's \cite{confluent_customers}, Confluent brings a wide variety of additional features to Kafka. This includes a user-friendly graphical interface for configuring and monitoring Kafka, a cloud service, a wide array of custom Kafka Connectors, and what they call 'Enterprise-grade Security'. A closer look shows that that encompasses secret protection (at-rest encryption of passwords, tokens, and configuration files), structured audit logs, as well as \ac{RBAC}. While providing robust security features, it does not address the need for anonymized access control within Kafka. \par
Confluent's Privitar Connector \cite{privitar} showcases that Confluent too recognizes the need for more anonymization granularity within Kafka. They intend to address that need for anonymization with their Privitar Connector. It provides basic anonymization techniques, i.e. Blurring and Suppression, for specified \ac{PII}. Their presentation at the Kafka Summit London \cite{confluent_privitar_summit} shows great potential for further anonymization granularity with more advanced anonymization techniques. However, as of the writing of this thesis, the development seems to still be in progress as indicated by their Connector Hub \cite{privitar}. Furthermore, its availability will be under a costly subscription, which contrasts with the open-source and cost-effective nature of the solution proposed in this thesis. \par
Our work in granular anonymization and access control for \acp{DES} is a novelty. By developing an open-source solution, we aim to foster access to privacy-enhancing tools without a paywall, potentially encouraging companies to invest more in user data privacy. We hope that other companies and researchers will build and expand on our methodologies, aiding us in our vision for the future with data protection as a fundamental aspect of \acp{DES}. 