\chapter{Related Work\label{cha:chapter6}}
Organizations worldwide are increasingly committed to complying with data privacy and protection regulations. This trend is also evident among users of Apache Kafka, who are adapting their systems to comply with these international standards. \par
Out-of-the-box Apache Kafka supports authentication with SASL, SSL, and Kerberos as well as authorization with \acp{ACL}, but only when specifically enabled. As highlighted throughout this thesis the default configuration of Kafka contains no inherent data privacy measures. Kafka's stance is clear: they leave enforcing data privacy regulations to their users. \par
In addressing this need, users of Kafka, such as Amazon AWS, have expanded upon the platform's basic authentication and authorization capabilities. Amazon AWS, for example, integrates multi-factor authentification and \ac{RBAC} on their clusters that run Kafka. Additionally, they encrypt all data flowing through Kafka \cite{aws_kafka_security}. Thus, Amazon understands the need for more privacy - but enacts it on its product instead of Kafka itself. This approach diverges from the core objective of this thesis, which is to embed such privacy features directly into distributed event stores. \par
There is one organization, however, that turns Apache Kafka into an enterprise data streaming platform. Founded by the creators of Apache Kafka, Confluent \cite{confluent} is built with Apache Kafka at heart. Utilized by prominent companies such as BMW, Bosch, the Deutsche Bahn, and Domino's \cite{confluent_customers}, Confluent brings a wide variety of additional features to Kafka. This includes a user-friendly graphical interface for configuring and monitoring Kafka, a cloud service, a wide array of custom Kafka Connectors, and what they call 'Enterprise-grade Security'. A closer look shows that that encompasses secret protection (at-rest encryption of passwords, tokens, and configuration files), structured audit logs, as well as \ac{RBAC}. While providing robust security features, it does not address the need for anonymized access control within Kafka. \par
Confluent's Privitar Connector \cite{privitar} showcases that Confluent too recognizes the need for more anonymization granularity within Kafka. The Privitar Connector provides basic anonymization techniques, i.e. Blurring and Suppression, for specified \ac{PII}. Their presentation at the Kafka Summit London \cite{confluent_privitar_summit} shows great potential for further anonymization granularity with more advanced anonymization techniques. However, as of the writing of this thesis, the development seems to still be in progress as indicated by their Connector Hub. Furthermore, its availability will be under a costly subscription, which contrasts with the open-source and cost-effective nature of the solution proposed in this thesis. \par
In the context of Kafka, our work in granular anonymization and access control is a novelty as far as we know. By developing an open-source solution, we aim to foster access to privacy-enhancing tools without a paywall, potentially encouraging companies to invest more in user data privacy.