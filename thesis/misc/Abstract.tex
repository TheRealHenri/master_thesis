Modern database systems are of incredible importance. Without them, our entire infrastructure would fall apart. Hospitals could no longer properly treat patients, lacking the necessary patient records and treatment history. Banks and financial services would be left unable to execute transactions. Telecommunication services would shut down. These modern database systems often utilize message brokers in combination with stream processing engines. They are capable of accumulating, processing, and distributing data in real-time at an unprecedented rate and volume. With the focus firmly on performance and scalability, data protection has been left behind. A critical oversight as protecting sensitive data is essential in ensuring personal privacy, preventing misuse and fraud, and upholding trust in data handling. At the moment, these database systems provide only very basic access control and lack mechanisms for enforcing privacy policies. Companies often resort to encrypting the data flowing through database systems and focus on external authentification and authorization. Processing encrypted data, however, comes with challenges including computational overhead, added complexity, and performance trade-offs. Decrypting the data again before processing leads back to square one. \par
This thesis introduces a novel model for data anonymization with integrated access control enforcing mechanisms uniquely within modern database systems, more specifically \acfp{DES}. We have realized our model through the development of a new system the \acf{DASH} for Apache Kafka, a leading \ac{DES}. \ac{DASH} is capable of applying a broad variety of anonymization techniques to data streams, uniquely within the database infrastructure. Our tests demonstrate \ac{DASH}'s ability to apply anonymization on individual tuples without introducing performance overhead. We found more complex anonymization techniques, such as those required for achieving k-anonymity on collections of tuples, to be strongly coupled with available system resources. Our evaluation reveals that simpler anonymization techniques are suited for even the highest performance demands, whereas more complex anonymization techniques are more limited in their application.